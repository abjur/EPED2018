\documentclass[12pt]{article}

\usepackage{ifxetex,ifluatex}

\ifnum 0\ifxetex 1\fi\ifluatex 1\fi=0 % if pdftex
  \usepackage[T1]{fontenc}
  \usepackage[utf8]{inputenc}
\else % if luatex or xelatex
  \ifxetex
    \usepackage{mathspec}
    \usepackage{xltxtra,xunicode}
  \else
    \usepackage{fontspec}
  \fi
  \defaultfontfeatures{Mapping=tex-text,Scale=MatchLowercase}
  \newcommand{\euro}{€}
\fi

\usepackage{amssymb,amsmath}
\usepackage{float}
\usepackage{fixltx2e} % provides \textsubscript
\usepackage{fancyhdr}
\usepackage[utf8]{inputenc}
\usepackage[T1]{fontenc}
\usepackage[portuguese]{babel}
\usepackage{graphicx}
\usepackage{indentfirst}
\usepackage{lipsum}
\usepackage{longtable}
\usepackage{multicol}
\usepackage[default,osfigures,scale=0.95]{opensans} %% Alternatively
%% use the option 'defaultsans' instead of 'default' to replace the
%% sans serif font only.
\usepackage[T1]{fontenc}
\usepackage{pdfpages}
\usepackage[colorlinks=true,urlcolor=blue,citecolor=blue,linkcolor=blue]{hyperref}
%\renewcommand{\familydefault}{\sfdefault}
\renewcommand{\baselinestretch}{1.5}

% Paper style
\usepackage[
  paperwidth=21cm,paperheight=29.7cm, % larger paper
  layoutwidth=21cm,layoutheight=27.82cm, % A4 landscape
  layouthoffset=0cm,layoutvoffset=1cm, % 1cm to crop on all sides
  left=2cm,right=2cm,top=2cm,bottom=2cm,
  headsep=\dimexpr3cm-72pt\relax,
  headheight=72pt]{geometry}

\pagestyle{fancy}
\lhead{}
\chead{}
\rhead{\includegraphics[scale = 1]{logo-abj.png}}
\lfoot{}
\cfoot{Associação Brasileira de Jurimetria \\ Rua Gomes de Carvalho, 1356, 2º andar. CEP 04547-005 - São Paulo, SP,
Brasil \\ \url{http://abj.org.br}}
\rfoot{\text{} \\ \text{} \\ \thepage}
\renewcommand{\headrulewidth}{0.4pt}
\renewcommand{\footrulewidth}{0.4pt}


% Header information
\title{Minicurso Jurimetria}
\author{Fernando Correia e José de Jesus Filho}

\providecommand{\tightlist}{%
  \setlength{\itemsep}{0pt}\setlength{\parskip}{0pt}}

\begin{document}

\maketitle

\thispagestyle{fancy}

{
\hypersetup{linkcolor=black}
\setcounter{tocdepth}{2}
\tableofcontents
}
\section{Ementa}\label{ementa}

Com os avanços da computação e o uso crescente da informática na
administração judiciária, existe disponível uma grande variedade de
dados para consulta pública pela internet. Essa abundância
informacional, entretanto, impõe uma série de desafios metodológicos.
Perguntas simples como ``Quais dados vou coletar?'', ``Como vou obter os
dados?'' e ``O que vou fazer com eles?'' fazem parte do cotidiano de
quem utiliza essas informações para pesquisa empírica. Neste workshop,
vamos tratar das principais formas com que a justiça organiza seus dados
e de quais formas eles podem ser acessados. Além disso, vamos (1)
mostrar como as ferramentas desenvolvidas pela Associação Brasileira de
Jurimetria podem ser usadas para obter dados de maneira eficaz e (2)
construir um panorama geral dos métodos usualmente aplicados em
pesquisas empíricas no Direito.

\section{Tópicos}\label{topicos}

\begin{enumerate}
\def\labelenumi{\arabic{enumi}.}
\item
  Introdução à pesquisa empírica no direito: estudos prospectivos e
  retrospectivo;
\item
  Quais dados vou coletar?

  \begin{itemize}
  \tightlist
  \item
    Onde se encontram dados do Judiciário: consulta processual,
    jurisprudência, Conselho Nacional de Justiça, diários de justiça
  \item
    Em que forma eles são armazenados?
  \item
    Como a minha pergunta de pesquisa se relaciona com os dados?
  \end{itemize}
\item
  Como obter os dados? (1h)

  \begin{itemize}
  \tightlist
  \item
    Ferramentas da ABJ para coleta de dados sobre o Judiciário
  \end{itemize}
\item
  O que vou fazer com os dados?
\end{enumerate}

\subsection{Estudo de casos}\label{estudo-de-casos}

\begin{enumerate}
\def\labelenumi{\arabic{enumi}.}
\tightlist
\item
  Coleta e análise de decisões sobre tráfico de drogas;
\item
  Coleta e análise de dados sobre audiência de custódia;
\item
  O tempo como resposta: especialização das varas empresariais;
\item
  O tempo do processo: análise da duração razoável do processo nas
  decisões do STF;
\item
  Extração de dados dos processos e das decisões judiciais: dados
  fáticos, elementos de provas, partes, substitutos processuais, perfil
  do réu, movimentação processo, dispositivo;
\item
  Deixando o texto falar com modelagem tópica.
\end{enumerate}

\section*{Bibliografia}\label{bibliografia}
\addcontentsline{toc}{section}{Bibliografia}

\hypertarget{refs}{}
\hypertarget{ref-ash2017}{}
Ash, Elliott, and Daniel Chen. 2017. ``Predicting Punitiveness from
Judicial Corpora.''

\hypertarget{ref-oxford2012}{}
Cane, Peter, and Herbert Kritzer. 2012. \emph{The Oxford Handbook of
Empirical Legal Research}. OUP Oxford.

\hypertarget{ref-epstein2014}{}
Epstein, Lee, and Andrew D Martin. 2014. \emph{An Introduction to
Empirical Legal Research}. Oxford University Press.

\hypertarget{ref-kastellec2010}{}
Kastellec, Jonathan P. 2010. ``The Statistical Analysis of Judicial
Decisions and Legal Rules with Classification Trees.'' \emph{Journal of
Empirical Legal Studies} 7 (2). Wiley Online Library: 202--30.

\hypertarget{ref-leiter1997}{}
Leiter, Brian. 1997. ``Rethinking Legal Realism: Toward a Naturalized
Jurisprudence.'' \emph{Tex. L. Rev.} 76. HeinOnline: 267.

\hypertarget{ref-liu2017}{}
Liu, Zhenyu, and Huanhuan Chen. 2017. ``A Predictive Performance
Comparison of Machine Learning Models for Judicial Cases.'' In
\emph{Computational Intelligence (Ssci), 2017 Ieee Symposium Series on},
1--6. IEEE.

\hypertarget{ref-zeisel2012}{}
Zeisel, Hans, and David Kaye. 2012. \emph{Prove It with Figures:
Empirical Methods in Law and Litigation}. Springer Science \& Business
Media.

\end{document}
